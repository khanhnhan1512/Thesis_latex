\chapter*{Tóm tắt}
\label{summary}

Trong bối cảnh dữ liệu tri thức ngày càng phát triển theo thời gian, Đồ thị Tri thức Thời gian (Temporal Knowledge Graphs - TKGs) đóng vai trò quan trọng trong việc biểu diễn các mối quan hệ động giữa các thực thể. Tuy nhiên, bài toán Suy luận Đồ thị Tri thức Thời gian (TKGR) vẫn đối mặt với nhiều thách thức: các phương pháp hiện tại chưa khai thác hiệu quả thông tin lịch sử bậc cao, gặp hạn chế trong xử lý khối lượng dữ liệu lớn và thiếu tính giải thích trong quá trình suy luận. Mặc dù sự xuất hiện của các Mô hình Ngôn ngữ Lớn (LLMs) mở ra tiềm năng mới, việc ứng dụng chúng vào TKGR vẫn bị cản trở bởi các chiến lược prompt đơn giản và thiếu khả năng tùy chỉnh theo truy vấn.

Nghiên cứu này đề xuất phương pháp MSKGen (Multi-Source Knowledge-Based Generation) - một cách tiếp cận mới kết hợp tri thức đa nguồn để giải quyết các hạn chế trên. Phương pháp của chúng tôi tích hợp ba thành phần chính: (1) Trích xuất quy tắc logic thời gian thông qua kỹ thuật temporal random walks và tinh chỉnh bằng LLMs, (2) Truy xuất tri thức ngữ nghĩa sử dụng cơ sở dữ liệu vector hóa kết hợp kỹ thuật RAG, và (3) Cơ chế lập luận đa nguồn hướng dẫn bởi truy vấn. Hệ thống đánh giá kết quả thông qua sự kết hợp giữa điểm số từ LLMs và mô hình đồ thị được huấn luyện trước.

Kết quả thực nghiệm trên các tập dữ liệu cho thấy MSKGen vượt trội so với các phương pháp tiên tiến hiện tại. Phương pháp này không chỉ nâng cao độ chính xác mà còn cung cấp khả năng giải thích quá trình suy luận thông qua việc kết hợp các quy tắc logic và tri thức ngữ nghĩa. Ứng dụng tiềm năng của MSKGen bao gồm dự đoán xu hướng kinh tế, hỗ trợ ra quyết định y tế và nâng cao hiệu suất của hệ thống khuyến nghị động, mở ra hướng nghiên cứu mới về tích hợp LLMs vào các bài toán đồ thị tri thức phức tạp.
