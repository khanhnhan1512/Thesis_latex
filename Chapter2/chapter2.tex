\chapter{CÁC CÔNG TRÌNH \\ LIÊN QUAN}
\label{Chapter2}

Chương này tập trung vào việc giới thiệu các mô hình tiên tiến cho bài toán dự đoán liên kết trong đồ thị tri thức thời gian, được phân loại thành hai nhánh giải quyết chính là nội suy và ngoại duy được giới thiệu ở chương 1. Để đảm bảo tính rõ ràng và thống nhất trong toàn bộ khóa luận cũng như các phần tiếp theo, Bảng 2.1 tóm tắt các ký hiệu được sử dụng. Nhiệm vụ dự đoán các thành phần bị thiếu trong đồ thị tri thức thời gian, cho dù chúng nằm trong hay ngoài tập dữ liệu hiện có, đã dẫn đến sự phân biệt hai phương pháp chính: Hoàn thiện dựa trên nội suy và
Hoàn thiện dựa trên ngoại suy. Các phương pháp này sẽ được trình bày ở các mục con tương ứng trong nội dung tiếp theo.


\section{Tóm tắt các ký hiệu và định nghĩa chung}

Trong nghiên cứu về suy luận đồ thị tri thức thời gian, việc thiết lập một hệ thống ký hiệu nhất quán và rõ ràng đóng vai trò quan trọng trong việc trình bày các khái niệm và phương pháp một cách chính xác. Hệ thống ký hiệu được sử dụng trong nghiên cứu này được xây dựng dựa trên các tiêu chuẩn quốc tế và phù hợp với các nghiên cứu tiền phong trong lĩnh vực đồ thị tri thức thời gian.

Đồ thị tri thức thời gian được định nghĩa là một chuỗi các snapshot có dấu thời gian \( G = \{G_1, G_2, ..., G_t, ...\} \), trong đó mỗi snapshot \( G_t = (\mathcal{E}, \mathcal{R}, \mathcal{T}) \) chứa các dữ kiện xảy ra tại thời điểm \( t \). Tập hợp \( \mathcal{E} \) đại diện cho tập thực thể, \( \mathcal{R} \) biểu thị tập quan hệ và \( \mathcal{T} \) là tập dấu thời gian. Nhiệm vụ dự đoán trong đồ thị tri thức thời gian nhằm dự đoán các thực thể thiếu trong các dấu thời gian tương lai.

Đối với một truy vấn \( q = (s_q, r_q, ?, t_q) \) hoặc \( q = (?, r_q, o_q, t_q) \), trong đó \( s_q, o_q \in \mathcal{E} \) là các thực thể chủ thể và đối tượng đã biết, \( r_q \in \mathcal{R} \) là quan hệ giữa chủ thể và đối tượng, \( t_q \in \mathcal{T} \) là dấu thời gian truy vấn và ``?'' biểu thị thực thể chưa biết. Mục tiêu là dự đoán thực thể thiếu bằng cách sử dụng chuỗi đồ thị tri thức thời gian \( G_{<t_q} = \{G_1, G_2, ..., G_{t_q-1}\} \).

Luật logic thời gian \cite{ref_article23} đóng vai trò quan trọng trong khung nghiên cứu này và được định nghĩa như công thức (1): 
\[
\rho := r(e_s, e_o, t_l) \Leftarrow \bigwedge_{i=1}^{l-1} r_i(e_s, e_o, t_i)
\]
trong đó phần bên trái biểu thị đầu luật với quan hệ \( r \) có thể được suy ra từ thân luật bên phải. Thân luật bao gồm một phép hội của các quan hệ \( r_i \) với các ràng buộc thời gian \( t_1 \leq t_2 \leq ... \leq t_{l-1} < t_l \).

\section{Hoàn thiện dựa trên nội suy}

Phương pháp nội suy trong toán học \cite{ref_article29} được áp dụng như một kỹ thuật quan trọng để ước tính giá trị của một hàm tại một điểm cụ thể dựa trên các giá trị đã biết. Trong bài toán hoàn thiện đồ thị tri thức thời gian, nội suy trở thành một phương pháp hiệu quả để dự đoán các dữ kiện bị thiếu trong phạm vi thời gian đã quan sát của đồ thị. Bằng cách phân tích các mẫu và mối quan hệ trong dữ liệu có giới hạn thời gian, các phương pháp nội suy suy luận các giá trị bị thiếu để cải thiện độ đầy đủ chung của biểu diễn đồ thị tri thức thời gian.

Phương pháp phân rã ten-xơ đã được chứng minh hiệu quả trong hoàn thiện đồ thị tri thức tĩnh và được ứng dụng thành công cho đồ thị tri thức thời gian. Kỹ thuật này tận dụng cấu trúc đặc trưng của đồ thị tri thức thời gian bằng cách mô hình hóa chúng dưới dạng các ten-xơ đa chiều. Mỗi chiều của tensor đại diện cho một thành phần riêng biệt: thực thể đầu, quan hệ, thực thể đuôi và trường thời gian. Bằng cách phân rã các ten-xơ bậc 4 này thành các ma trận có chiều thấp hơn, phương pháp này học được các biểu diễn nén gọn phản ánh chính xác sự phát triển động của tri thức theo thời gian.

Trong số các phương pháp triển khai, các mô hình như TComplEx \cite{ref_article31} sử dụng biểu diễn phức để mô hình hóa đồ thị tri thức thời gian. Mô hình này tích hợp thông tin thời gian bằng cách liên kết dữ kiện thời gian với thực thể và quan hệ thông qua các bản nhúng thời gian phức tạp. Các phương pháp tiếp theo như TNTComplEx \cite{ref_article31} cải tiến bằng cách giới thiệu các thành phần phi thời gian, giúp giảm thiểu tác động của thành phần thời gian đến độ tin cậy của mô hình.

Phương pháp chuyển vị quan hệ thời gian tận dụng các kỹ thuật hoàn thiện đồ thị tri thức tĩnh hiện có bằng cách kết hợp thêm thông tin thời gian. Các mô hình thuộc phương pháp này xem xét thành phần thời gian như phép biến đổi giữa các thực thể có tích hợp thêm thời gian. Phát triển từ TransE \cite{ref_article32}, các mô hình như TTransE \cite{ref_article32} trực tiếp tích hợp thông tin thời gian bằng cách kết hợp thông tin thời gian với quan hệ, tạo ra các quan hệ phụ thuộc thời gian tổng hợp.

Các mạng nơ-ron tiên tiến ngày càng được ứng dụng nhiều trong đồ thị tri thức thời gian và mang lại hiệu suất cao trên các tập dữ liệu chuẩn. Các phương pháp này tận dụng những kiến trúc mạng nơ-ron khác nhau, bao gồm mạng nơ-ron hồi quy, mạng nơ-ron đồ thị và mạng nơ-ron tích chập. Mỗi kiến trúc được thiết kế để giải quyết những thách thức đặc thù của đồ thị tri thức thời gian một cách hiệu quả, như việc nắm bắt các phụ thuộc thời gian dài hạn và ngắn hạn trong cấu trúc đồ thị động.

\section{Hoàn thiện dựa trên ngoại suy}

Áp dụng nguyên tắc ngoại suy trong toán học \cite{ref_article30}, các mô hình hoàn thiện đồ thị tri thức thời gian tập trung vào việc dự đoán các dữ kiện chưa được quan sát, mở rộng phạm vi dự đoán vượt ra ngoài giới hạn thời gian đã biết. Những mô hình này phân tích kỹ lưỡng các mẫu và xu hướng phát triển được ghi nhận trong các snapshot lịch sử của đồ thị tri thức thời gian. Quá trình phân tích này cho phép mô hình học các biểu diễn chính xác cho các thực thể và quan hệ, phản ánh sự biến đổi động của đồ thị tri thức theo thời gian.

\textbf{Phương pháp học sâu (Deep Learning) cho ngoại suy đồ thị tri thức thời gian} đã có những bước tiến vượt bậc trong những năm gần đây, đặc biệt là với sự xuất hiện của các mô hình mạng nơ-ron đồ thị (GNNs). Các phương pháp tiêu biểu bao gồm:

\begin{itemize}
    \item \textbf{RE-NET} \cite{ref_article12}: Sử dụng mạng nơ-ron hồi quy (RNN) để mã hóa chuỗi các sự kiện trong quá khứ, kết hợp với module tổng hợp láng giềng dựa trên Relational GCN để nắm bắt đồng thời phụ thuộc cấu trúc cục bộ và phụ thuộc chuỗi thời gian dài hạn. RE-NET xây dựng mô hình tự hồi quy để dự đoán xác suất xuất hiện của các sự kiện tương lai dựa trên toàn bộ lịch sử các sự kiện trước đó.
    \item \textbf{RE-GCN} \cite{ref_article13}: Mô hình này kết hợp GCN nhận biết quan hệ để học các biểu diễn tiến hóa của thực thể và quan hệ tại mỗi thời điểm, đồng thời sử dụng thành phần hồi quy để mô hình hóa các mẫu tuần tự trên toàn bộ chuỗi lịch sử. RE-GCN còn bổ sung ràng buộc tĩnh cho thực thể để cải thiện chất lượng biểu diễn, giúp mô hình dự đoán hiệu quả các sự kiện tương lai với tốc độ vượt trội so với các phương pháp trước đó.
    \item \textbf{TANGO} \cite{ref_article33}: Mở rộng các phương pháp trên bằng cách tích hợp các phương trình vi phân thường (neural ODEs) vào mạng tích chập đồ thị đa quan hệ, cho phép xử lý thông tin thời gian mịn hơn và mô hình hóa động lực học liên tục của các quan hệ trong đồ thị tri thức thời gian.
    \item \textbf{HGLS} \cite{ref_article14}: Đề xuất kiến trúc phân cấp, kết hợp cả phụ thuộc thời gian dài hạn và ngắn hạn thông qua thiết kế mạng nơ-ron đồ thị mới, khắc phục hạn chế của các phương pháp trước trong việc nắm bắt toàn diện các mẫu thời gian phức tạp.
    \item \textbf{TiRGN} \cite{ref_article22}: Đề xuất mô hình mạng nơ-ron đồ thị hồi quy có hướng dẫn thời gian với các mẫu lịch sử cục bộ-toàn cục (Time-Guided Recurrent Graph Network with Local-Global Historical Patterns). TiRGN đồng thời nắm bắt ba đặc trưng lịch sử quan trọng: mẫu tuần tự (sequential), mẫu lặp lại (repetitive), và mẫu chu kỳ (cyclical) của các sự kiện lịch sử. Mô hình sử dụng bộ mã hóa hồi quy cục bộ với cơ chế hồi quy kép để mô hình hóa phụ thuộc lịch sử của các sự kiện tại các mốc thời gian liền kề, kết hợp với bộ mã hóa lịch sử toàn cục để thu thập các sự kiện lặp lại trong toàn bộ lịch sử. TiRGN tích hợp vector thời gian chu kỳ và phi chu kỳ vào bộ giải mã để nắm bắt tính chu kỳ của các sự kiện, đồng thời thiết kế cơ chế cân bằng giữa thông tin lịch sử cục bộ và toàn cục thông qua hệ số trọng số có thể điều chỉnh.
\end{itemize}

Các phương pháp học sâu này đã chứng minh hiệu quả vượt trội trên nhiều bộ dữ liệu chuẩn, giúp mô hình hóa tốt các phụ thuộc cấu trúc và thời gian trong đồ thị tri thức động, từ đó nâng cao khả năng dự đoán các sự kiện chưa từng xuất hiện trong lịch sử.

\textbf{Phương pháp suy tập luật thời gian (temporal rule-based reasoning)} xuất phát từ thành công trong bài toán hoàn thiện đồ thị tri thức tĩnh với tính giải thích và độ tin cậy cao. Khi sử dụng phương pháp này cho bài toán hoàn thiện đồ thị tri thức thời gian, các trường thời gian được tận dụng làm thông tin quan trọng cho các luật logic \cite{ref_article23}, được biểu diễn dưới dạng 
\[
H \Leftarrow B_1 \wedge B_2 \wedge ... \wedge B_n
\]
trong đó $H$ là đầu luật được suy ra nếu các điều kiện $B_i$ được thỏa mãn. Cấu trúc này phản ánh cách lập trình logic ngược và bao gồm các phụ thuộc thời gian giữa các thực thể.

Các mô hình sử dụng hướng giải quyết này học hiệu quả các luật logic thời gian thông qua việc tận dụng độ tương đồng giữa các bản nhúng đường dẫn và nhúng quan hệ. Tận dụng độ tương đồng cosin giữa các bản nhúng đường dẫn và nhúng quan hệ, ALRE-IR \cite{ref_article34} cung cấp một mô hình nhúng thich nghi luật logic có khả năng trích xuất các đường dẫn quan hệ từ các lát cắt của đồ thị tri thức thời gian và đánh giá độ tin cậy của các luật này. TLogic \cite{ref_article23} sử dụng các bước ngẫu nhiên để trích xuất các luật logic thời gian tuần hoàn, cho phép nắm bắt các phụ thuộc qua các mốc thời gian khác nhau và đưa ra các lời giải thích hợp lý để con người hiểu và nắm bắt dễ dàng. Cuối cùng, ILR-IR \cite{ref_article35} kết hợp các phương pháp nhúng và quy tắc dựa trên luật, nắm bắt logic nhân quả một cách kỹ lưỡng thông qua việc học các nhúng luật và các tương tác ưu tiên giữa chúng. Những phương pháp này cung cấp khả năng trích xuất các đường dẫn quan hệ từ các snapshot của đồ thị tri thức thời gian và đánh giá độ tin cậy của các luật. Suy tập luật thời gian cung cấp một cách tiếp cận hấp dẫn cho hoàn thiện dựa trên ngoại suy bằng cách cung cấp các luật có thể được giải thích, nắm bắt tốt các phụ thuộc thời gian phức tạp.


\textbf{Ứng dụng các mô hình ngôn ngữ lớn (LLMs) trong suy luận đồ thị tri thức thời gian} đã mở ra những hướng tiếp cận mới đầy tiềm năng. Các nỗ lực ban đầu tập trung vào việc áp dụng trực tiếp LLMs thông qua học trong ngữ cảnh, với các phương pháp như GPT-NeoX-ICL \cite{ref_article08} thể hiện tiềm năng của dự đoán với số lượng mẫu nhỏ thông qua kỹ thuật prompt engineering cẩn thận. Chain-of-History \cite{ref_article09} tiến bộ hơn bằng cách giới thiệu phương pháp suy luận từng bước để khai thác thông tin lịch sử bậc cao, giải quyết hạn chế của việc xử lý khối lượng lớn thông tin lịch sử cùng một lúc.

Các phát triển gần đây bao gồm GenTKG \cite{ref_article10}, giới thiệu một framework retrieval-augmented generation mới kết hợp truy xuất dựa trên quy tắc logic thời gian với tinh chỉnh tham số hiệu quả với số lượng mẫu nhỏ. Tuy nhiên, các ứng dụng LLM hiện tại trong suy luận đồ thị tri thức thời gian vẫn gặp phải những thách thức về việc khai thác không đầy đủ khả năng hiểu ngữ nghĩa của LLMs trong các tác vụ TKGR. Việc sử dụng các phương pháp truy xuất truyền thống chỉ dựa vào lọc dữ kiện thông qua khớp schema dẫn đến việc khai thác hạn chế thông tin lịch sử và thiếu suy luận theo truy vấn cụ thể.


\section{Phương pháp lựa chọn}

Mặc dù có nhiều kỹ thuật khác nhau cho cả hoàn thiện đồ thị tri thức dựa trên nội suy và ngoại suy, việc phân tích toàn diện mỗi phương pháp đòi hỏi sự cân nhắc kỹ lưỡng về các yêu cầu cụ thể và khả năng thực hiện của nhiệm vụ đang được thực hiện. Mỗi phương pháp đều có ưu điểm và hạn chế riêng, tạo nên sự đa dạng trong lựa chọn giải pháp tùy thuộc vào bối cảnh ứng dụng và đặc thù của dữ liệu.

Do nhu cầu về hiệu quả và linh hoạt trong việc xử lý dữ liệu phức tạp cũng như khả năng dự đoán các sự kiện trong tương lai, nghiên cứu này lựa chọn phương pháp tiếp cận ngoại suy để giải quyết bài toán suy luận đồ thị tri thức thời gian. Quyết định này xuất phát từ nhu cầu thực tế về khả năng dự báo và suy luận về các sự kiện có thể xảy ra trong tương lai dựa trên thông tin lịch sử, một yêu cầu quan trọng trong nhiều ứng dụng thực tế.

Việc tích hợp các mô hình ngôn ngữ lớn mang lại những lợi thế đáng kể trong việc xử lý thông tin ngữ nghĩa và khả năng suy luận phức tạp. LLMs thể hiện khả năng xuất sắc trong việc hiểu các mối liên kết ngữ nghĩa và các mẫu thời gian trong dữ liệu, đồng thời cung cấp giải pháp tiềm năng cho các thách thức về tính minh bạch trong suy luận đồ thị tri thức thời gian. Khả năng này đặc biệt quan trọng khi cần xử lý các truy vấn cụ thể và tạo ra các câu trả lời có tính thích ứng cao.

Mô hình MSKGen được phát triển nhằm tối ưu hóa việc kết hợp nhiều nguồn tri thức khác nhau để tạo ra các dự đoán chính xác và có độ tin cậy cao. Bằng cách tích hợp dữ kiện dựa trên luật với dữ kiện được truy xuất ngữ nghĩa, MSKGen duy trì tính minh bạch đồng thời tối đa hóa khả năng ngữ nghĩa của LLMs. Phương pháp này giải quyết các thách thức về tải thông tin mà các ứng dụng LLM hiện tại gặp phải và mang lại những tiến bộ đáng kể trong việc kết hợp suy luận thời gian có cấu trúc với khả năng hiểu ngữ nghĩa cho các tác vụ suy luận đồ thị tri thức.